\documentclass[12pt,a4paper]{book}
\usepackage[utf8]{inputenc}
%\usepackage[spanish, es-noquoting]{babel}
\usepackage[left=2.5cm,right=2.5cm,top=2.5cm,bottom=2.5cm]{geometry}
\usepackage{amsmath}
\usepackage{amsfonts}
\usepackage{amssymb}
\usepackage{amsthm, mathtools}
\usepackage{tikz,tikz-cd}
\usetikzlibrary{arrows, babel}
\usepackage{url}
\urlstyle{rm}
\usepackage[colorlinks=true,linktocpage=true,pagebackref=true,linkcolor=blue,urlcolor=blue]{hyperref}
\usepackage{graphicx}
%\usepackage{tocstyle}
%\usepackage{tocbibind}
%\usepackage{mathrsfs}
\usepackage[scr=boondox]{mathalfa}
\usepackage{anyfontsize}

%Fuente Palatino:
%\usepackage[sc]{mathpazo}
%Fuente Times:
\usepackage{newtxtext}
\usepackage{newtxmath}
%Fuente Libertine:
%\usepackage{libertine}
%\usepackage[libertine]{newtxmath}

\newtheorem{thm}{Theorem}[section]
\newtheorem{prop}[thm]{Proposition}
\newtheorem{lema}{Lemma}
\newtheorem{corol}[thm]{Corollary}
\theoremstyle{definition} \newtheorem{defn}[thm]{Definition}
\theoremstyle{definition} \newtheorem{ejemplo}[thm]{Example}
\theoremstyle{remark} \newtheorem*{rem}{Remark}

\def\pr{\mathrm{pr}}
\def\OO{\mathscr{O}}
\def\AA{\mathscr{A}}
\def\MM{\mathscr{M}}
\def\GG{\mathscr{G}}
\def\CC{\mathbf{C}}
\def\gg{\mathfrak{g}}
\def\ad{\mathrm{ad}}
\def\ZZ{\mathbf{Z}}
\def\RR{\mathbf{R}}
\def\KK{\mathbf{K}}
\def\SF{\mathbf{S}}
\def\TT{\mathbf{T}}
\def\NN{\mathbf{N}}
\def\HH{\mathbf{H}}
\def\PP{\mathbf{P}}
\def\tr{\mathrm{tr}}
\def\rk{\mathrm{rk}}
\def\id{\mathbf{1}}
\def\im{\mathrm{im}\ }
\def\End{\mathrm{End}}
\def\Aut{\mathrm{Aut}}
\def\Hom{\mathrm{Hom}}
\def\Sym{\mathrm{Sym}}
\def\Pic{\mathrm{Pic}}
\def\Spec{\mathrm{Spec}}
\def\Span{\mathrm{span}}
\def\delbar{\bar{\partial}}
\newcommand{\ve}[1]{\mathbf{#1}}

\let\emph\relax
\DeclareTextFontCommand{\emph}{\it\bfseries}

\DeclarePairedDelimiter\esc{\langle}{\rangle}
\DeclarePairedDelimiter\norm{\lVert}{\rVert}
\DeclarePairedDelimiter\abs{\lvert}{\rvert}

%%%        Otro formato para las secciones
\usepackage{titlesec}
\usepackage{remreset}

\renewcommand{\thesection}{\arabic{section}}
\renewcommand{\thechapter}{\Roman{chapter}}

\titleformat{\chapter}[block]
{\fontsize{20}{22}\bfseries\sffamily}
{\hfill CHAPTER \thechapter}
{1em}
{\vspace{1cm} \hrule\vspace{.5cm}\hfill}
[\vspace{.5cm}\hrule]


\titleformat{\section}[block]
{\fontsize{15}{18}\bfseries\sffamily\filcenter}
{\S\ \thesection.}
{1em}
{}

%\makeatletter
%\@removefromreset{section}{chapter}
%\makeatother
%%%

\title{\bfseries \sffamily Higgs bundles twisted by a vector bundle}
\author{\it Guillermo Gallego Sánchez}
\date{}


\begin{document}
\maketitle
\tableofcontents
\chapter{Vector bundles on Riemann surfaces}
\section{Topological classification of vector bundles}
The first step towards the classification of vector bundles on Riemann surfaces is their ``topological'' classification. That is, we want to classify smooth complex vector bundles on a Riemann surface up to $C^\infty$ isomorphism. This is indeed pretty easy to do, since the problem can be reduced to the classification of line bundles.
\begin{thm}
  If $E$ is a rank $n$ smooth complex vector bundle over a compact Riemann surface $X$ then it is isomorphic to $\det E \oplus (X\times \CC^{n-1})$.
\end{thm}
\begin{proof}
   We will proceed by induction on $n$. Of course, if $E$ is a line bundle, $E\cong \det E$. Now, let $n>1$, suppose that it is true for any vector bundle of rank $n-1$ and let $E$ be a rank $n$ vector bundle.
   \begin{lema}
     $E$ has a nowhere vanishing section.
   \end{lema}
   \begin{proof}
     Let $s_0$ be the zero section of $E$ and $S_0=s_0(X)\subset E$. By the transversality theorem [REF: Hirsch], we can densely choose a section $s\in \Gamma(E)$ transversal to $S_0$. Now, if $s$ vanishes at some point $x\in X$, then $s(x)\in S_0$ and, since $s$ is transversal to $S_0$, $$d_xs(T_xX) + T_{s(x)}S_0 = T_{s(x)}E.$$ But $\dim_\RR E=2n+2$, $\dim_\RR S_0 = 2$ and $\dim_\RR d_xs(T_xX) \leq \dim_\RR X = 2$, so, if $n>1$, dimensions do not add up to verify the above equality and therefore we get a contradiction. 
   \end{proof}

   Let us continue with the proof of the theorem. Since $E$ has a nowhere vanishing section $s$, we can define the line bundle
   \begin{equation*}
     L=\bigsqcup_{x\in X} \Span(s(x)),
   \end{equation*}
   with 
   \begin{align*}
      \pi:L&\longrightarrow X\\ 
       \lambda s(x) &\longmapsto x.
     \end{align*}
     Therefore, we can decompose $E=E'\oplus L$, with $E'$ a rank $n-1$ vector bundle. For example, fixing a metric on $E$, we can define $E'$ to be the orthogonal complement of $L$. But observe now that $L$ is isomorphic to the trivial line bundle: the bundle morphism
     \begin{align*}
      X\times \CC  &\longrightarrow L\\ 
	 (x,\lambda) &\longmapsto \lambda s(x), 
       \end{align*}
       is in fact an isomorphism. This can be proven by defining a metric on $L$ and normalizing $s \mapsto s/\norm{s}$. Then we can define the inverse
     $ y\in L \mapsto (\pi(y), \esc{s(\pi(y)),y})\in X\times \CC$. 

     Thus, we have shown that $E\cong E'\oplus (X\times \CC)$. Now, applying the induction hypothesis, $E'\cong \det E' \oplus (X\times \CC^{n-2})$, so $E\cong \det E' \oplus (X\times \CC^{n-1})$. Finally, via transition functions it can be easily shown that $\det E \cong \det E'$.
\end{proof}

This last theorem says that vector bundles can be topologically classified by their rank and their determinant, which is a line bundle. Let us proceed then with the classification of line bundles. Recall that all the data of a vector bundle can be recovered by the transition functions $\{g_{\alpha \beta}\in C^\infty(U_\alpha \cap U_\beta , \CC)\}$ defining it, where $\left\{ U_\alpha \right\}$ is an open cover of $X$. These functions verify the cocycle condition $$g_{\alpha \beta}=g_{\gamma \beta}\cdot g_{\alpha \gamma}.$$ Also recall that an isomorphism of vector bundles induces a coboundary on the transition functions
\begin{equation*}
  \tilde{g}_{\alpha\beta}=f^{-1}_{\delta\beta}\cdot g_{\gamma\delta} \cdot f_{\gamma\alpha}.
\end{equation*}
Therefore, topological ($C^\infty$) isomorphism classes of vector bundles are parametrized by the \v{C}ech cohomology group
\begin{equation*}
  H^1(X,C^{\infty,*}_X).
\end{equation*}
To obtain more information about this cohomology group we are going to introduce a very powerful tool: the first Chern class of a vector bundle.

Recall from Chern-Weil theory [REF:Griffiths-Harris,Wells] that for any complex vector bundle $E$ over $X$ and for any connection $\nabla$ on $E$ with associated curvature form $F$, the $2$-form $\tr F$ is closed and its de Rham cohomology class $[\tr F]\in H_{\mathrm{dR}}^2(X)$ does not depend on the choice of the connection, so it is an invariant of the vector bundle $E$. If we normalize this form to get an integer cohomology class, we define the \emph{first Chern class} of $E$ as the cohomology class:
\begin{equation*}
  c_1(E)=\left[ \frac{i}{2\pi} \tr F \right] \in H_{\mathrm{dR}}^2(X).
\end{equation*}
We define the \emph{degree} of a vector bundle $E$ as the pairing of $c_1(E)$ with the fundamental class of $X$, that is
\begin{equation*}
  \deg E = \int_X\frac{i}{2\pi} \tr F. 
\end{equation*}
The next proposition [REF:Wells] summarizes the most important properties about the degree that we are going to use:
\begin{prop}[Propierties of the degree]
  Let $E$ and $F$ be complex vector bundles over a compact Riemann surface $X$.
  \begin{enumerate}
    \item $\deg(E)$ depends only on the isomorphism class of $E$.
    \item $\deg(E) \in \mathbf{Z}$.
    \item $\deg(E\oplus F)=\deg(E)+ \deg(F)$.
    \item $\deg(E\otimes F)=\rk F \deg E + \rk E \deg F.$
    \item $\deg E=\deg(\det E) $.
    \item
      Let $\delta:H^1(X,C^{\infty,*}_X) \rightarrow H^2(X,\ZZ)$ be the connecting homomorphism of the long exact sequence in cohomology induced by the exponential sheaf exact sequence
      \begin{center}
	\begin{tikzcd}
	  0 \arrow{r} &	  \ZZ \arrow{r} & C^\infty_X \arrow{r}{\exp} & C^{\infty,*}_X \arrow{r} &0,
	 \end{tikzcd}
       \end{center}
       where $\exp(f)=e^{2\pi if}$.
      The diagram
      \begin{center}
	\begin{tikzcd}
	  H^1(X,C^{\infty,*}_X)	  \arrow{r}{\delta}\arrow{rd}[anchor=north,rotate=-30]{c_1} & H^2(X,\ZZ) \arrow[hook]{d}\\ 
	  & H^2_{\mathrm{dR}}(X),
	 \end{tikzcd}
       \end{center}
       is commutative.
  \end{enumerate}
\end{prop}

This last property will be crucial in the classification of line bundles. Let us consider again the exponential sheaf exact sequence
      \begin{center}
	\begin{tikzcd}
	  0 \arrow{r} &	  \ZZ \arrow{r} & C^\infty_X \arrow{r}{\exp} & C^{\infty,*}_X \arrow{r} & 0.
	 \end{tikzcd}
       \end{center}
       The existence of smooth partitions of unity implies that the sheaf $C^\infty_X$ is fine, so $H^1(X,C^\infty_X)=H^2(X,C^\infty_X)=0$. Therefore, the connecting map $\delta:H^1(X,C^{\infty,*}_X)\rightarrow H^2(X,\ZZ)$ is an isomorphism. If we now consider the set of second de Rham cohomology classes with integer coefficients $H^2_{\mathrm{dR}}(X,\ZZ)$, which is just the image of $H^2(X,\ZZ)$ by the inclusion $H^2(X,\ZZ)\hookrightarrow H^2_{\mathrm{dR}}(X)$, we get an isomorphism $c_1:H^1(X,C_X^{\infty,*})\rightarrow H^2_{\mathrm{dR}}(X,\ZZ)$. Now, the isomorphism $H^2_{\mathrm{dR}}(X) \cong \CC$ given by integration on $X$, descends to an isomorphism $H^2_{\mathrm{dR}}(X,\ZZ) \cong \ZZ$.
    Summarizing, we have the diagram
       \begin{center}
	 \begin{tikzcd}
	   H^1(X,C_X^{\infty,*}) \arrow{r}{c_1} \arrow[bend right]{rr}{\deg}& H^2_{\mathrm{dR}}(X,\ZZ) \arrow{r}{\int_X} & \ZZ.
	 \end{tikzcd}
       \end{center}
       That is, the degree gives an isomorphism between the set isomorphism classes of smooth line bundles and $\ZZ$. This concludes the topological classification of vector bundles, which we can gather in the next theorem
       \begin{thm}
	 Smooth complex vector bundles over a compact Riemann surfaces are classified, up to $C^\infty$ isomorphism by their rank and their degree.
       \end{thm}
  
       \section{The problem of classification of holomorphic vector bundles}
       Now that we have classified vector bundles up to topological ($C^\infty$) isomorphism, we are going to pursue the full classification of holomorphic vector bundles over Riemann surfaces. According to the results of last section, we can reduce our problem to the study of the ``list'' of isomorphism classes of holomorphic vector bundles of fixed rank $n$ and degree $d$ (regarding them in particular as complex vector bundles). From the beginning, this problem gets really involved, since this ``lists'' are so big that they themselves admit a geometric structure. These are the so called \emph{moduli spaces}. Therefore, the classification problem translates to that of investigating the geometric properties of the associated moduli spaces.

       To illustrate these ideas in more detail, let us consider the case of holomorphic line bundles. The same arguments regarding \v{C}ech cocycles of the previous section also apply now to show that the isomorphism classes of holomorphic vector bundles are parametrized by the sheaf cohomology group
       \begin{equation*}
	 H^1(X,\OO_X^*).
       \end{equation*}
       This cohomology group is called the \emph{Picard group} of $X$ and we denote it by $\Pic(X)$.
       Now, as before, we can consider the exponential sheaf sequence
      \begin{center}
	\begin{tikzcd}
	  0\arrow{r}&	  \ZZ \arrow{r} & \OO_X \arrow{r}{\exp} & \OO_X^* \arrow{r} & 0,
	 \end{tikzcd}
       \end{center}
       and the connecting operator $\delta:H^1(X,\OO_X^*)\rightarrow H^2(X,\ZZ)$ of the induced long exact sequence in cohomology. Analogously to the previous section, one can show that the diagram 
      \begin{center}
	\begin{tikzcd}
	  \Pic(X)	  \arrow{r}{\delta}\arrow{rd}[anchor=north,rotate=-30]{c_1} & H^2(X,\ZZ) \arrow[hook]{d}\\ 
	  & H^2_{\mathrm{dR}}(X)
	 \end{tikzcd}
       \end{center}
       is commutative. However, unlike the smooth case, the sheaf $\OO_X$ is not fine, since there are not holomorphic partitions of unity, so in general $\delta$ is not an isomorphism anymore. Let $\Pic^0(X)=\ker \delta \subset \Pic(X)$ be the subgroup of degree zero line bundles. Then, we have an exact sequence
      \begin{center}
	\begin{tikzcd}
	  0\arrow{r}&	  H^1(X,\ZZ) \arrow{r} & H^1(X,\OO_X) \arrow{r} & \Pic^0(X) \arrow{r}{\delta} & 0.
	 \end{tikzcd}
       \end{center}
       Therefore $\Pic^0(X)\cong H^1(X,\OO_X)/H^1(X,\ZZ)$. Now, the Dolbeaut theorem says that $H^1(X,\OO_X)\cong H^{0,1}(X)$ and the Hodge decomposition theorem says that $H^1(X,\CC)\cong H^{1,0}(X) \oplus H^{0,1}(X)$. Also, by Serre duality, $H^{1,0}(X)\cong H^{0,1}(X)^*$ and via Mayer-Vietoris one can easily prove that $H^1(X,\CC)=\CC^{2g}$ and that $H^1(X,\ZZ)=\ZZ^{2g}$, where $g$ is the genus of $X$. Putting all this together we have that $H^1(X,\OO_X)\cong \CC^{g}$, so $\Pic^0(X)$ is isomorphic to the complex torus
       \begin{equation*}
J(X):=	 \frac{H^1(X,\OO_X)}{H^1(X,\ZZ)} \cong \frac{\CC^g}{\ZZ^{2g}}.
       \end{equation*}
       We call this complex torus the \emph{Jacobian} of $X$. The other components of fixed degree of $\Pic(X)$ can be also shown to be isomorphic to the Jacobian of $X$, via the isomorphism
       \begin{align*}
	  \Pic^d(X)&\longrightarrow \Pic^0(X)\\ 
	   L &\longmapsto L\otimes M, 
	 \end{align*}
	 where $M \in \Pic^{-d}(X)$ is a fixed line bundle. It is of course injective, since if $L\otimes M = L'\otimes M$, then 
	 \begin{equation*}
	   L=L\otimes M \otimes M^* = L' \otimes M \otimes M^* = L',
	 \end{equation*}
	 and it is also surjective since for every line bundle $L \in \Pic^0(X)$, $L=(L\otimes M^*) \otimes M$.
       
	 The result for line bundles already gives some hints on the complexity of the general problem. However, for very low genus the problem can be solved relatively easy. For genus 0, Grothendieck proved that every holomorphic vector bundle on the Riemann sphere $\PP^1_\CC$ can be decomposed as a direct sum of line bundles. A proof of this result can be found in [REF:Hartshorne]. For genus 1, it was Atiyah who showed that the ``moduli space'' of indecomposable vector bundles with fixed rank and degree over an elliptic curve is isomorphic to the curve itself. Check out [REF:Mukai] for a construction of the moduli space and for a proof of this. 

	 The problem gets its full complexity in the case of general genus $g\geq 2$, for which we will dedicate the rest of the chapter. To construct a ``good'' moduli space (one with nice topological properties, like being Hausdorff) we need to consider the idea of stability, arising from Mumford's Geometric Invariant Theory [REF:GIT]. This theory also allows to construct the moduli space of stable vector bundles, although it can also be done using analytic methods [REF:Kobayashi]. This moduli space has very nice and interesting geometric properties and it has been studied from the algebraic point of view (for example in the works of Narasimhan, Ramanan or Seshadri) as well as from the analytical or gauge-theoretical point of view (by Atiyah, Bott, Donaldson or Hitchin, for example).

	 \section{Holomorphic structures as Dolbeaut operators}
	 Let $\ve{E}$ be a holomorphic vector bundle on a compact Riemann surface $X$. Associated to $\ve{E}$, we have the \emph{Dolbeaut operator}
	 \begin{align*}
	   \delbar_\ve{E}:\Omega^{p,q}(X,\ve{E})&\longrightarrow \Omega^{p,q+1}(X,\ve{E})
	   \end{align*}
	   which satisfies
	   \begin{equation*}
	     \delbar_{\ve{E}}(\alpha\psi)=(\delbar \alpha)\psi + (-1)^{p} \alpha \wedge \delbar_{\ve{E}} \psi,
	   \end{equation*}
	   for every $\alpha \in \Omega^p(X)$, $\psi \in \Gamma(\ve{E})$, and 
	   \begin{equation*}
	     \delbar_{\ve{E}}^2=0.
	   \end{equation*}

	   Conversely, if $E$ is a complex vector bundle, any operator $\delbar_{\ve{E}}:\Omega^{p,q}(X,E)\rightarrow \Omega^{p,q+1}(X,E)$ which satisfies the conditions above induces on $E$ the structure of a holomorphic vector bundle. The idea here is that $\delbar_{\ve{E}}^2=0$ is the \emph{integrability condition} for the PDE $\delbar_{\ve{E}} s=0$ (check out [REF:Donaldson4Variedades] for a proof of this fact). The solutions of this PDE are the holomorphic vector functions, so the sheaf of local solutions is locally free over $\OO_X$ and therefore it is an holomorphic vector bundle supported on $E$. In conclusion, if we define $\AA_{\delbar}$ as the set of all such $\delbar_{\ve{E}}$ operators satisfying the previous conditions, this set is in bijection with the set of all holomorphic structures on $E$.

	   Now let us consider the group $\GG^c$ of \emph{gauge transformations} of a complex vector bundle $E$, that is, diffeomorphisms $g:E\rightarrow E$ such that the diagram
	   \begin{center}
	     \begin{tikzcd}
	       E \arrow{rr}{g} \arrow{rd} && E \arrow{ld}	       \\ 
	       & X &
	     \end{tikzcd}
	   \end{center}
	   commutes. That is, $\GG^c=\Gamma(\Aut(E))$. This group acts on $\AA_{\delbar}$ by the rule
	   \begin{equation*}
	     g\cdot \delbar_{\ve{E}}=g\delbar_{\ve{E}} g^{-1}.
	   \end{equation*}
	   Therefore we can identify the quotient set $\AA_{\delbar}/\GG^c$ with the set of isomorphism classes of holomorphic vector bundles of rank $\rk E$ and degree $\deg E$. Even though $\AA_{\delbar}$ and $\GG^c$ are infinite-dimensional spaces, using analytic techniques [REF:Kobayashi] they can be given the structure of \emph{Banach manifolds} (roughly, spaces locally modeled by Banach spaces) and this quotient space can be precisely constructed. This has however a serious problem: the obtained space is not Hausdorff. Nevertheless, if we restrict ourselves to the class of \emph{stable vector bundles}, with the same analytic methods we can obtain a ``good'' moduli space.
	   \begin{defn}
	     Let $E$ be a complex vector bundle over a compact Riemann surface $X$. We define the \emph{slope} of $E$ as the number
	     \begin{equation*}
	       \mu(E)=\deg E/ \rk E.
	     \end{equation*}
	     We say that a holomorphic vector bundle $\ve{E}=(E,\delbar_{\ve{E}})$ is \emph{stable} if for every holomorphic subbundle $\ve{E}'\subset \ve{E}$ (that is, for every subbundle $E'\subset E$ such that $\delbar$ preserves $E'$), $$\mu(E')<\mu(E).$$ 
	     Let $n=\rk E$ and $d=\deg E$. We consider $\AA_{\delbar}^s\subset \AA_{\delbar}$ the subset of stable holomorphic bundles $\ve{E}=(E,\delbar_{\ve{E}})$ and define the \emph{moduli space of stable holomorphic vector bundles} of rank $n$ and degree $d$ as the quotient
	     \begin{equation*}
	       M_{n,d}=\AA_{\delbar}^s/\GG^c.
	     \end{equation*}
	   \end{defn}
	   \begin{thm}
	     The moduli space $M_{n,d}$ is a complex manifold of dimension $1+n^2(g-1)$, where $g$ is the genus of $X$.
	   \end{thm}

	   \section{Holomorphic structures and unitary connections}
	   Although we will not enter into detail, the main reason why the quotient $\AA_{\delbar}/\GG^c$ is not Hausdorff and why we need to introduce the stability condition is that the group $\GG^c=\Gamma(\Aut(E))$ is a complex group. Indeed, it is the complexification of the group $\GG=\Gamma(\Aut(E,h))$, where $h$ is a Hermitian metric on $E$. The idea for this is essentially that the general linear group $\mathrm{GL}(n,\CC)$ is the complexification of the unitary group $\mathrm{U}(n)$. This motivates the study of unitary connections,
which will give a gauge-theoretical approach to holomorphic vector bundles. This will allow us to give another analytical construction of the moduli space, this time as the space of solutions (up to gauge equivalence) to some differential equation.

Let $\ve{E}=(E,\delbar_{\ve{E}})$ be a holomorphic vector bundle on $X$ and let $h$ be a Hermitian metric on E. Recall that for any connection $\nabla:\Gamma(E) \rightarrow \Omega^1(X,E)=\Omega^{1,0}(X,E) \oplus \Omega^{0,1}(X,E)$ on $E$ there is a natural splitting $\nabla=\nabla^{1,0}+\nabla^{0,1}$, where
\begin{align*}
  \nabla^{1,0}:\Gamma(E) \longrightarrow \Omega^{1,0}(X,E), \\
  \nabla^{0,1}:\Gamma(E) \longrightarrow \Omega^{0,1}(X,E).
\end{align*}
The \emph{Chern connection} in $(\ve{E},h)$ is the unique $h$-\emph{unitary} connection (that is, $$d\esc{\xi,\eta}=\esc{\nabla \xi,\eta} + \esc{\xi,\nabla \eta},$$ for $\xi, \eta$ local sections of $E$) such that $$\nabla^{0,1}= \delbar_{\ve{E}}:\Gamma(E) \rightarrow \Omega^{0,1}(X,E).$$ Go to [REF: Wells] for a proof of the existence and uniqueness of this connection. 

Conversely, given any $h$-unitary connection $\nabla$ on $E$, we can define a holomorphic structure by fixing $\delbar_{\ve{E}}=\nabla^{0,1}$ and extending to operators $\Omega^{p,q}(X,E) \rightarrow \Omega^{p,q+1}(X,E)$ by linearity. Therefore, the set holomorphic structures on $E$ can be identified with the set $\AA_h$ of all $h$-unitary connections on $E$.

Note now that for any connection on $E$, is curvature must satisfy that the cohomology class
\begin{equation*}
  [\tr F] = -i2\pi c_1(E).
\end{equation*}
Fixing an area form $\omega_X$ on $X$ so that $\int_X \omega_X=1$, we can choose a representative of $c_1(E)$ of the form $k\omega_X$, where $k\in \CC$ is a constant. Of course,
\begin{equation*}
  \deg E = \int_X c_1(E) = \int_X C\omega_X=k,
\end{equation*}
so $k=\deg E$. Therefore we can ask if there is a connection on $E$ such that its curvature satisfies
\begin{equation*}
  \tr F = -2\pi i\deg(E) \omega_X.
\end{equation*}
Or, more generally, if $\id_{E}$ denotes the identity endomorphism of $E$, we can ask whether 
\begin{equation*}
  F = -2\pi i\frac{\deg E}{\rk E} \id_E \omega_X.
\end{equation*}
Recall that we defined the number $\mu(E)=\deg E /\rk E$ as the slope of $E$. 

\begin{defn}
  We say that a connection $\nabla$ on a complex vector bundle $E$ has \emph{constant central curvature} if is curvature $F$ satisfies
  \begin{equation*}
    F=-2\pi i \mu(E) \id_E \omega_X.
  \end{equation*}
  In particular, if $\deg E=0$, $F=0$ and we say that $\nabla$ is a \emph{flat connection}.
\end{defn}

Finally, we want to consider connections that are \textit{irreducible}:
\begin{defn}
  A unitary connection $\nabla$ on a complex Hermitian vector bundle $(E,h)$ is \emph{reducible} if $(E,h)=(E_1,h_1)\oplus (E_2,h_2)$ and $\nabla=\nabla_1\oplus \nabla_2$. We say that $\nabla$ is \emph{irreducible} if it is not reducible.
\end{defn}

Let us consider then the set $\AA_h^s$ of all $h$-unitary irreducible connections of central constant curvature on $E$. The gauge group $\GG$ acts on connections by conjugation $\nabla \mapsto g\nabla g^{-1}$, $g\in \GG$, and this actions preserves irreducibility and the equation of constant central curvature, so the group $\GG$ acts on $\AA_h^s$. Now, the same analytic techniques mentioned in the previous section [REF:Kobayashi] allow us to construct a ``good'' quotient:
\begin{thm}
  The moduli space of irreducible constant central curvature unitary connections $\AA_h^s/\GG$ on $(E,h)$ has the structure of a smooth real manifold of dimension $2+2n^2(g-1)$, where $n=\rk E$ and $g$ is the genus of $X$.
\end{thm}

Now, Donaldson's version of the theorem of Narasimhan-Seshadri relates this moduli space with the moduli space of stable holomorphic vector bundles.

\begin{thm}[Narasimhan-Seshadri]
  Let $(E,h)$ be a Hermitian complex vector bundle of rank $n$ and degree $d$ on a compact Riemann surface $X$. An irreducible unitary connection $\nabla$ has constant central curvature if and only if the associated holomorphic vector bundle $(E,\nabla^{0,1})$ is stable.
\end{thm}
This can be reformulated in terms of moduli spaces:
\begin{corol}
  The map
  \begin{align*}
    \AA_h&\longrightarrow \AA_{\delbar}\\ 
     \nabla &\longmapsto \nabla^{0,1}, 
    \end{align*}
    descends to a homeomorphism
    \begin{equation*}
      \AA_h^s/\GG \cong \AA^s_{\delbar} /\GG^c = M_{n,d}.
    \end{equation*}
\end{corol}
We will prove the ``easy'' direction of the equivalence. The proof in the other direction consists in defining a Yang-Mills functional and look for a minimum of it using analytic techniques, in particular a theorem by Uhlenbeck [REF:Uhlenbeck]. Check [REF:Donaldson] for the details.
\begin{proof}
First, let us suppose that $\nabla$ has constant central curvature
\begin{equation*}
  F=-2\pi i \frac{d}{n} \id_E \omega_X,
\end{equation*}
and define $\delbar_{\ve{E}}=\nabla^{0,1}$. Let $E'\subset E$ be a subbundle preserved by $\delbar_{\ve{E}}$. The Hermitian metric gives a smooth splitting
\begin{equation*}
  E=E'\oplus E'',
\end{equation*}
and we can write
\begin{equation*}
  \delbar_{\ve{E}}=\left(
  \begin{array}{cc}
    \delbar_{\ve{E}'} & \beta \\
    0 & \delbar_{\ve{E}''}
  \end{array}\right),
\end{equation*}
where $\delbar_{\ve{E}'}$ and $\delbar_{\ve{E}''}$ are the restrictions of $\delbar_{\ve{E}}$ to $E'$ and $E''$ and $\beta \in \Omega^{0,1}(X,\Hom (E'',E'))$. Now $\nabla$ can be written as
\begin{equation*}
  \nabla=\left(
  \begin{array}{cc}
    \nabla_{E'} & \beta \\
    -\beta^\dagger & \nabla_{E''}
  \end{array}\right),
\end{equation*}
where $\nabla_{E'}$ and $\nabla_{E''}$ are the connections associated to $\delbar_{\ve{E}'}$ and $\delbar_{\ve{E}''}$ and $\beta^\dagger=\star \bar{\beta} \in \Omega^{1,0}(X,\Hom(E',E''))$ is the transpose (on the matrix part) conjugate (on the form part) of $\beta$. The curvature of $\nabla$ can be written now as
\begin{equation*}
  F=\left(
  \begin{array}{cc}
    F_{E'}-\beta\wedge \beta^\dagger & \nabla_{\Hom(E'',E')}\beta \\
    -\nabla_{\Hom (E',E'')}\beta^\dagger & F_{E''}-\beta^\dagger \wedge \beta
  \end{array}\right)=-2\pi i\frac{d}{n} \id_E \omega_X.
\end{equation*}
The first corner of this equality now says that
\begin{equation*}
  F_{E'}-\beta \wedge \beta^\dagger = -2\pi i \frac{d}{n} \id_{E'} \omega_X.
\end{equation*}
Taking the trace and integrating we get
\begin{equation*}
  \frac{i}{2\pi}\int_X \tr F_{E'} - \frac{i}{2\pi} \int_X \tr(\beta \wedge \beta^\dagger) = d\frac{\rk E'}{n}.
\end{equation*}
Therefore
\begin{equation*}
  \frac{\deg E'}{ \rk E'} = \frac{d}{n} + \frac{i}{2\pi} \int_X \tr(\beta \wedge \beta^\dagger).
\end{equation*}
Now, $\tr(\beta\wedge \beta^\dagger)$ is precisely the Hodge pairing $\tr(\beta \wedge \star \bar{\beta}) = - \norm{\beta}^2 \omega_X$. Thus we have proven that 
\begin{equation*}
\mu(E')= \mu(E) - \norm{\beta}^2 .
\end{equation*}
The connection $\nabla$ is irreducible, so $\norm{\beta}\neq 0$ and $\mu(E')< \mu(E)$. That is, $E$ is stable.
\end{proof}
\end{document}
